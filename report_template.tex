\documentclass[12pt]{article}               
\usepackage[left=2cm, right=2cm, bottom=2cm, top=3cm]{geometry}
\geometry{letterpaper} 
\usepackage{cite}
\usepackage{lipsum}
\usepackage{amssymb,amsmath}
\usepackage{epstopdf}
\usepackage{setspace}
\usepackage{amsthm}
\usepackage{tikz}
\usepackage{tocloft}
%\usepackage{fancyhdr}
\usepackage{float} % place image here
\usepackage{pdflscape,lipsum} % landscape page
\usepackage[absolute]{textpos}
\usepackage{pdfpages,titleps}
\usepackage{enumitem}
\usepackage[utf8]{inputenc}
\usepackage{graphicx}
\usepackage[framed,numbered,autolinebreaks]{mcode} % code blocks
\usepackage[font=small,skip=0pt]{caption} % gap between table and caption
\usepackage[margin=1cm]{caption} % captioon distance from margines
\captionsetup[table]{skip=7pt} % table caption
\captionsetup[subfigure]{aboveskip=10pt,belowskip=8pt} % distance of subfigures caption from main caption

\usepackage{amsmath} % matrix in math
\usepackage{mathtools} % bold math equation 
\usepackage[numbers]{natbib}
\usepackage{caption}
\usepackage{subcaption}
\usepackage{color}   % may be necessary if you want to color links
\usepackage{hyperref}
\hypersetup{
	colorlinks=true,
	citecolor=blue, %set true if you want colored links
	linktoc=all,     %set to all if you want both sections and subsections linked
	linkcolor=blue,  %choose some color if you want links to stand out
}

\usepackage[nottoc,numbib]{tocbibind} % refrences in TOC
\renewcommand{\cftsecleader}{\cftdotfill{\cftdotsep}} % adding dots for sections to the page number in TOC

\makeatletter
\newcommand*{\rom}[1]{\expandafter\@slowromancap\romannumeral #1@}
\makeatother % roman numbers

\newcommand*{\fullref}[1]{\hyperref[{#1}]{\nameref*{#1}}} % hyperlink to section name


% customizing the header of the pages , each part can be customized(left,center,right)
\newpagestyle{main}{ 
	\setheadrule{0.5pt}
	\sethead{\thesection.~\sectiontitle}  % controling left side of the header 
	{} % center of the header, if you want to put anything like name: {name}
	{\subsectiontitle\quad$|$\quad\thepage} % right part of the header: |number
}
\pagestyle{main}
\renewcommand{\makeheadrule}{\rule[-0.5\baselineskip]{\linewidth}{0.1pt}} % space of the header line to above, default is: -0.5

\usepackage{indentfirst}
\setlength{\parindent}{15pt} %indentation at the first lines of each paragraph

\newcommand{\code}[1]{\texttt{#1}}
%%%%%%%%%%%%%%%%%%%%%%%%%%%%%%%%%%%%

\title{Report Template in LaTex\\\vspace{10pt} \small\textit{Implementation}\vspace{0.5ex}}

\normalsize
\author{Alireza Sedghi}
\date{\small{\today}}

%%%%%%%%%%%%%%%%%%%%%%%%%%%%%%%%%%%%%%

\begin{document}
%no numbering on title page
\maketitle
\thispagestyle{empty} 

%table of content
\newpage
\tableofcontents
\thispagestyle{empty}

% list of figures nad tables
\newpage
\listoffigures
\listoftables
\pagenumbering{roman}

%starting numbering
\newpage
\pagenumbering{arabic}

%%%%%%%%%%%%%%% Abstract %%%%%%%%%%%%%%%%%%%%%%%%%%
%using section* for having as a section but not numbering it
\section*{Abstract}
\addcontentsline{toc}{section}{Abstract} % putting it in the toc, because of section*{} we need to do this
\thispagestyle{plain} % not having the header on abstract page

%remove lipsum and put your text here, lipsum is for generating random text
This is a template for a report written in Latex, I tried my best to comment all the packages I used, and the reason I used them. You are more than welcome to contribute to this template. I did not wrote all of it by myself, but I put together pieces of the code to come up with this template. I will explain details of how you can customize it for your need. putting the Abstract in \code{\textbackslash section\{\}} will make it a section with number, but putting it in \code{\textbackslash section*\{\}} give us unnumbered section. this cause a problem of not being listed in table of content(TOC), we will solve it by forcing TOC to have a abstract section with \code{\textbackslash addcontentsline\{toc\}\{section\}\{Abstract\}}



%%%%%%%%%%%%% Introduction %%%%%%%%%%%%%%%%%%%%%%%%%
\newpage
\section{Introduction}

\lipsum[3]Referencing \cite{ref1} to have a reference example. \lipsum[3]\lipsum[1] In \autoref{fig:1} you can see latex logo and in \autoref{fig:2} it is also latex.
\lipsum[2-4]

\begin{figure}[h!]
	\centering
	\begin{minipage}[t]{.5\textwidth}
		\centering
		\includegraphics[width=1\linewidth]{pic/Picture1.png}
		\caption{picture 1}
		\label{fig:1}
	\end{minipage}%
	\begin{minipage}[t]{.5\textwidth}
		\centering
		\includegraphics[width=0.5\linewidth]{pic/Picture2.png}
		\caption{picture 2}
		\label{fig:2}
	\end{minipage}
\end{figure}

\subsection{a new subsection}
\lipsum[3]

%%%%%%%%%%%%% methodology %%%%%%%%%%%%%%%%%%%%%%%%%
\newpage
\section{methodology}
\par
\lipsum[3]
\noindent
Without indentation paragraph, Referencing \cite{ref2} to have a reference example. In \autoref{table:1} you can see a table \footnote{wow}.

\begin{table}[ht]
	\centering
	\caption{caption}
	\label{table:1}
	\begin{tabular}{|l|l|}
		\hline
		\textbf{name}                   & type1          \\ \hline
		\textbf{type}                   & mmmmmmm        \\ \hline
		\textbf{and method}             & min            \\ \hline
		\textbf{or method}              & max            \\ \hline
		\textbf{newnewnew method}       & centroid       \\ \hline
		\textbf{input}                  & 5              \\ \hline
		\textbf{output}                 & 6              \\ \hline
		\textbf{rules}                  & 17             \\ \hline
	\end{tabular}
\end{table}

\begin{itemize}
	\item
	\textbf{004.nrrd:} a sample CT image of scoliosis patient (anonymized due to preserving confidential information)
	\item
	\textbf{047.nrrd:} a sample CT image of scoliosis patient (anonymized due to preserving confidential information)
\end{itemize}

\begin{lstlisting}
% customizing the header of the pages , each part can be customized(left,center,right)
\newpagestyle{main}{ 
\setheadrule{0.5pt}
\sethead{\thesection.~\sectiontitle}  % controling left side of the header 
{} % center of the header, if you want to put anything like name: {name}
{\subsectiontitle\quad$|$\quad\thepage} % right part of the header: |number
}
\pagestyle{main}
\renewcommand{\makeheadrule}{\rule[-0.5\baselineskip]{\linewidth}{0.1pt}} % space of the header line to above, default is: -0.5
\end{lstlisting}

%%%%%%%%%%%%% References %%%%%%%%%%%%%%%%%%%%%%%%%

\newpage

\bibliographystyle{unsrtnat}
\bibliography{ref}


\end{document}